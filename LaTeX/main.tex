\RequirePackage{plautopatch}
\RequirePackage[l2tabu, orthodox]{nag}

% \documentclass[platex,dvipdfmx]{jlreq}			% for platex
\documentclass[platex,dvipdfmx, twocolumn]{jlreq}			% for platex
% \documentclass[uplatex,dvipdfmx]{jlreq}		% for uplatex
\usepackage{graphicx}
\usepackage{bxtexlogo}

\title{予稿}

\author{東海大学理学部物理学科 伊與田研究室 9BSP1118 村岡海人}
\date{\today}
\begin{document}
\maketitle
\section{目的}
量子ビットとは量子情報の最小単位である。
量子ビットの数だけ計算を同時に行うことができる。
その結果、一部の問題については量子コンピュータの方が我々が使っている古典コンピュータよりも高速であることが証明されている。
この「問題」を解くために用いられるのがアルゴリズムであり、量子コンピュータで使われるアルゴリズムを古典コンピュータと区別するため、量子アルゴリズムと呼ぶ。
量子アルゴリズムとは、ユニタリ演算子をたくさん掛けることである。
ユニタリーとは、ハミルトニアンを使って$exp(-iht)$で表すことができる。

\section{背景}
fugaufga

\section{基本事項}

\section{結果}
% \begin{figure}
% \centering
% \includegraphics[width=70mm]{figures/Sample.png}
% \caption{ここにキャプションを挿入します}
% \label{fig:model}
% \end{figure}
\end{document}