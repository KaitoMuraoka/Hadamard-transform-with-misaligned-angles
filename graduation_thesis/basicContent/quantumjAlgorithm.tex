\begin{comment}
    量子アルゴリズムについての記述
    24
\end{comment}

量子コンピュータは、量子力学的な重ね合わせによって、$n$個の量子ビットを用いて$2^n$個の状態を同時に処理できる。
しかし、これだけでは「計算が速い」ということにはならない。
なぜなら、計算終了後に結果を観測する際に、$2^n$個の状態の内どれか1つがランダムに得られるのみだからである。
したがって、欲しい答えが高確率で得られるように設計された、量子コンピュータ専用のアルゴリズムが不可欠である。
そのようなアルゴリズムを量子アルゴリズムと呼ぶ\cite{QuantumDojo}。


% 量子アルゴリズムの有名な例として、Shorの量子フーリエ変換と、本研究の対象であるグローバーのアルゴリズムがある。
% Shorの量子フーリエ変換は素因数分解や離散対数問題を解くアルゴリズムに基づいている。
% 最良の古典アルゴリズムに比べて指数関数的な著しい高速化を実現する。
% グローバーのアルゴリズムは

量子アルゴリズムのクラスは主に2つ存在する。
最初のクラスは素因数分解や離散対数問題を解くアルゴリズムを含む、Shorの量子フーリエ変換に基づくものであり、最良の古典アルゴリズムに比べて指数関数的な著しい高速化を実現する。
2つ目のアルゴリズムのクラスは量子探索を行うグローバーのアルゴリズムに基づくものである。
これはそれほど著しくないが、それでも最良の古典アルゴリズムに比べて、2乗の顕著な高速化を実現する。
量子探索アルゴリズムはが重要なのは古典アルゴリズムで探索ベースの技術が広く使われており、多くの事例で古典アルゴリズムを高速の量子アルゴリズムに直接変更することができるからである\cite{QuantumComputation&Infomation2}。