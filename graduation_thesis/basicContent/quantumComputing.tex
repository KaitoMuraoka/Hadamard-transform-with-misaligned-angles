\begin{comment}
    量子計算についての記述
\end{comment}

\subsubsection{単一量子ビットゲート}
量子ビットに対する論理ゲートを説明するために、まず古典コンピュータについて考える。
古典コンピュータ回路は配線と論理ゲートより構成される。
配線は回路中の情報を運び、論理ゲートは情報を変換して操作する。
例えば、古典的単一ビット論理ゲートであるNOTゲートを考える。 
% 真理値表を作成するように
NOTゲートの働きは表に示す真理値表で定義され、$0 \rightarrow 1$及び$1 \rightarrow 0$つまり、状態$0, 1$の入れ替えが行われる。
真理値表とは入力と出力の対応を表すための表である。

量子ビットに対しても同様にNOTゲートを定義することができる。
量子ビットへの操作は線形変換であり、単一量子ビットに作用する量子ゲートは$2 \times 2$の行列で記述できる。
特に量子ビットに対するNOTゲートは次のように定義される行列で表される。

\begin{equation}
    X = \begin{pmatrix}
        0 & 1\\
        1 & 0
    \end{pmatrix}
\end{equation}
この行列のことをXゲートと呼ぶ。
また、量子状態$\alpha \ket{0} + \beta \ket{1}$をベクトル表記で次のように記述する。
\begin{equation}
    \alpha \ket{0} + \beta \ket{1} = \begin{pmatrix}
        \alpha \\
        \beta
    \end{pmatrix}
\end{equation}
ここで、上の要素は$\ket{0}$に対する振幅、下の要素は$\ket{1}$に対する振幅に相当する。
量子NOTゲートの出力は、
\begin{equation}
    X \begin{pmatrix}
        \alpha \\
        \beta
    \end{pmatrix}
    = \begin{pmatrix}
        \beta \\
        \alpha
    \end{pmatrix}
\end{equation}
で与えられる。
このXゲートはブラケット表記で次のようにも記述できる。
\begin{equation}
    X = \ket{0}\bra{1} + \ket{1}\bra{0}
\end{equation}

量子ゲートの行列に対する制約について議論する。
量子状態$\alpha \ket{0} + \beta \ket{1}$に対して正規化条件$|\alpha|^2 + |\beta|^2 = 1$が必要であることを考慮すると、正規化条件はゲート作用後の量子状態$\ket{\psi'} = \alpha' \ket{0} + \beta' \ket{1}$に対しても成立しなければならない。
この条件を満たすのは、単一量子ビットゲートを記述する行列$U$がユニタリ、つまり$U^{\dagger}U = I$を満たす場合である。
ここで、$U^{\dagger}$は$U$のエルミート共役、Iは$2 \times 2$の単位行列である。

単一量子ビットに対する量子ゲートとして、後で用いる3つのゲートを導入する。
Zゲートは、
\begin{equation}
    Z = \begin{pmatrix}
        1 & 0 \\
        0 & -1
    \end{pmatrix}
    = \ket{0}\bra{0} - \ket{1}\bra{1}
\end{equation}
と定義される。
Yゲートは、
\begin{equation}
    Y = \begin{pmatrix}
        0 & -i \\
        i & 0
    \end{pmatrix}
    = -i \ket{1}\bra{0} - i \ket{0}\ket{1}
\end{equation}
と定義される。
アダマールゲートは、
\begin{equation}
    H \equiv \frac{1}{\sqrt{2}}\begin{pmatrix}
        1 & 1 \\
        1 & -1
    \end{pmatrix}
    = \frac{1}{\sqrt{2}}(\ket{0}\bra{0} + \ket{1}\bra{0} + \ket{0}\bra{1} - \ket{1}\bra{1})
\end{equation}
と定義される。
アダマールゲートを$\ket{0}$または$\ket{1}$に作用させると、
\begin{equation}
    H\ket{0} = \frac{1}{\sqrt{2}}(\ket{0} + \ket{1})
\end{equation}
\begin{equation}
    H \ket{1} = \frac{1}{\sqrt{2}}(\ket{0} - \ket{1})
\end{equation}
に変換される。
これらは簡単な代数計算で、$X^2 = Y^2 = Z^2 = H^2 = I$とわかる。

\subsubsection{1量子ビットの任意の回転}
% \subsubsection{任意角度のスピンと1量子ビットの状態}
任意のユニタリ回転ゲートを作成する。
$x, y, z$軸周りに1量子ビットを回転させる行列は、次のようにパウリ演算子$X, Y, Z$から求められる。
\begin{equation}
    R_x(\theta) = e^{-i \theta X / 2} = \cos{\frac{\theta}{2}}I - i \sin{\frac{\theta}{2}} X = \begin{pmatrix}
        \cos{\frac{\theta}{2}} & -i \sin{\frac{\theta}{2}} \\
        -i \sin{\frac{\theta}{2}} & \cos{\frac{\theta}{2}} \\
    \end{pmatrix}
\end{equation}

\begin{equation}
    R_y(\theta) = e^{-i \theta Y / 2} = \cos{\frac{\theta}{2}} I - i \sin{\frac{\theta}{2}} Y =  \begin{pmatrix}
        \cos{\frac{\theta}{2}} & -i\sin{\frac{\theta}{2}} \\
        \sin{\frac{\theta}{2}} & \cos{\frac{\theta}{2}} \\
    \end{pmatrix}
\end{equation}

\begin{equation}
    R_z(\theta) = e^{-i \theta Z / 2} = \cos{\frac{\theta}{2}}I - i \sin{\frac{\theta}{2}} Y = \begin{pmatrix}
        e^{-i \theta / 2} & 0 \\
        0 & e^{i\theta/2}
    \end{pmatrix}
\end{equation}
そして、1量子ビットの任意のユンタリ回転ゲートは、これらの$Z-Y$回転行列で分解できる。
実数$\gamma, \phi, \theta, \lambda$を用いて、
\begin{equation}
    \label{eq: global}
    U(\theta, \phi, \lambda) = e^{i \gamma} R_z(\phi) R_y(\theta) R_z(\lambda) 
    = e^{i(\gamma - \frac{\theta}{2} - \frac{\phi}{2})}
    \begin{pmatrix}
        \cos{\frac{\theta}{2}} & -e^{i\lambda} \sin{\frac{\theta}{2}} \\
        e^{i\phi}\sin{\frac{\theta}{2}} & e^{i(\lambda + \phi) \cos{\frac{\theta}{2}}}
    \end{pmatrix}
\end{equation}
上記で用いられる$e^{i(\gamma - \frac{\theta}{2} - \frac{\phi}{2})}$は、全体位相と呼ばれ、ブロッホ球上の回転操作や回転角に直接関わることなく、また実際に観測される量ではないため、実際の任意の回転行列は、
\begin{equation}
    \label{eq: hoge}
    U(\theta, \phi, \lambda) = 
    \begin{pmatrix}
        \cos{\frac{\theta}{2}} & -e^{i \lambda}\sin{\frac{\theta}{2}} \\
        e^{i \phi} \sin{\frac{\theta}{2}} & e^{i(\lambda + \phi)} \cos{\frac{\theta}{2}}
    \end{pmatrix}
\end{equation}
となる。

% このユニタリ回転ゲートを用いいると、アダマール演算子$H$は、
% \begin{align}
%     H &= \frac{1}{\sqrt{2}} \begin{pmatrix}
%         1 & 1 \\
%         1 & -1
%     \end{pmatrix} \\
%     &= \begin{pmatrix}
%         \cos{\frac{\pi}{4}} & -e^{i \pi} \sin{\frac{\pi}{4}} \\
%         e^{i 0}\sin{\frac{\pi}{4}} & e^{i \pi} \cos{\frac{\pi}{4}}
%     \end{pmatrix} \\
%     &= U(\frac{1}{2}\pi, 0, \pi)
% \end{align}
% となる。

\subsubsection{多量子ビットゲート}
複数の量子ビットからなる多量子ビットについての量子ゲートについて考える。
まず、2量子ビットに作用する制御NOTゲートを説明する。
このゲートは、制御量子ビットと標的量子ビットに作用し、もし制御量子ビットの状態が$\ket{0}$ならば標的量子ビットには何もせず、$\ket{1}$ならば標的量子ビットにNOTゲートを作用させる。
% 式で表すと、
\begin{equation}
    \ket{00} \rightarrow \ket{00}, 
    \ket{01} \rightarrow \ket{01},
    \ket{10} \rightarrow \ket{11},
    \ket{11} \rightarrow \ket{10}
\end{equation}
となる。
制御NOTゲートに対する回路表現を図に示す。
上の線が制御量子ビット、下の線が標的量子ビットを表している。
回路図については〇〇節で説明する。
制御NOTゲートの作用は、$A, B = 0, 1$とした場合に、$\ket{A, B} \rightarrow \ket{A, B \oplus A}$と書くことができる。
ここで、$\oplus$は2を法とする和を表す。
また、制御NOTゲートは
\begin{equation}
    U_{CN} = \begin{pmatrix}
        1 & 0 & 0 & 0 \\
        0 & 1 & 0 & 0 \\
        0 & 0 & 0 & 1 \\
        0 & 0 & 1 & 0 \\
    \end{pmatrix}
\end{equation}

\subsubsection{量子回路}
古典の回路図を拡張した量子回路図(図〇〇)を用いて説明する。
回路図は左から右に向けて読む。
回路図の各線は量子回路の1本の配線を表す。
この配線は必ずしも物理的な線に対応するものではなく、時間経過に対応したり、空間のある場所から別の場所に移動する光子のような物理的粒子に対応することもある。
また、制御Uゲートの回路図を図〇〇に示す。
ここでUは$n$個の量子ビットに作用するユニタリ演算子であり、n個の量子ビットの量子ゲートと見なせる。
このとき、制御NOTゲートの拡張として制御Uゲートを定義する。
このゲートは黒点を伴う線で表される単一の制御量子ビットと、箱に入ったUで示されるn個の標的量子ビットよりなっている。
図〇〇では6個の標的ビットよりなっている
もし、制御量子ビットの状態が$\ket{0}$ならば標的量子ビットには何も起きない。
もし、制御量子ビットの状態が$\ket{1}$ならば標的量子ビットに対してゲートUが作用する。
また、制御NOTゲートは図〇〇に示すように$U-X$とおいた制御Uゲートに相当する。