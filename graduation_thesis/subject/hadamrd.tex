\begin{comment}
    ここでは任意の回転ゲートにおけるアダマール演算子について記述する。
\end{comment}
ここでは任意の回転ゲートにおけるアダマール演算子について記述する。

グローバーのアルゴリズムで初期状態を用意するために使用した
アダマール演算子$H$は、式\ref{eq: hoge}を用いて$Z-Y$回転行列に分解すると、
\begin{equation}
    \begin{split}
        H &= \frac{1}{\sqrt{2}} \begin{pmatrix}
            1 & 1 \\
            1 & -1
        \end{pmatrix} \\
        &= \begin{pmatrix}
            \cos{\frac{\pi}{4}} & -e^{i\pi} \sin{\frac{\pi}{4}} \\
            e^{i0} \sin{\frac{\pi}{4}} & e^{i \pi} \cos{\frac{\pi}{4}}
        \end{pmatrix} \\
        &= U(\frac{1}{2}\pi, 0, \pi)
    \end{split}
\end{equation}
と表すことができる。

本研究では、このアダマール演算子$H$に$y$軸方向のズレ$\delta_y$, $z$軸方向のズレ$\delta_z$が生じている場合を考え、

\begin{itemize}
    \item 初期状態に$y$軸方向のズレ$\delta_y$を加えたグローバーのアルゴリズム
    \item 初期状態に$z$軸方向のズレ$\delta_z$を加えたグローバーのアルゴリズム
    \item 初期状態に$y$軸方向、$z$軸方向のズレ$\delta_y, \delta_z$を加えたグローバーのアルゴリズム
\end{itemize}

の3つの場合を量子ビット数を増やして数値計算を行う。


