\RequirePackage{plautopatch}
\RequirePackage[l2tabu, orthodox]{nag}

\documentclass[platex,dvipdfmx, titlepage]{jlreq}			% for platex
% \documentclass[uplatex,dvipdfmx]{jlreq}		% for uplatex
\usepackage{graphicx}
\usepackage{bxtexlogo}
\usepackage{braket} % ブラケット
\usepackage{amsmath} % 行列
\usepackage{comment} % コメント
\usepackage{tikz}
\usepackage{braket}

\title{初期状態が不完全なグローバーのアルゴリズムの振る舞いについて}

\author{9BSP1118 村岡海人}
\date{\today}
\begin{document}
\maketitle
\tableofcontents
\clearpage

\makeatletter
\renewcommand{\theequation}{%
\thesection.\arabic{equation}}
\@addtoreset{equation}{section}
\makeatother

\section{はじめに}
% \input overview.tex

\subsection{研究の背景}
量子コンピュータとは、量子力学を利用して計算を行うコンピュータである。
% 量子コンピュータについてもっと詳しく書く
この量子コンピュータで行う計算を量子計算と呼び、量子計算におけるアルゴリズムのことを量子アルゴリズムと呼ぶ。
例えば、多項式時間で整数を因数分解するショアのアルゴリズムや、整列化されていないデータベースからデータベースから特定のデータを探索するグローバーのアルゴリズムがある。


\subsection{研究の目的}
従来の計算機が論理演算から構成されているのと同様、量子計算も量子演算から構成されており、この量子演算は、時間に依存するシュレディンガー方程式から記述することができる。
量子計算を行う際に、ハミルトニアンや時間がズレてしまうと実現したい操作からズレた操作を行うことになり、アルゴリズム自体の出力に対す るエラーになってしまう。
本研究では、初期状態を準備する操作が不完全な場合に、グローバーのアルゴリズムがどれだけ機能するか調べることを目的とした。
% \subsection{本論文の構成}
% 大部分が書き終えてから

\section{基本的内容}
\input basicInfomation.tex



% \begin{figure}
% \centering
% \includegraphics[width=70mm]{figures/Sample.png}
% \caption{ここにキャプションを挿入します}
% \label{fig:model}
% \end{figure}


\end{document}