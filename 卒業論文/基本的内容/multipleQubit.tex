\begin{comment}
    多量子ビット
\end{comment}


多量子ビットの状態について考えてみる。
簡単のため、2個の量子ビットがあるとする。
これが古典ビットならば4つの取り得る状態$00, 01, 10, 11$がある。
これに対して2個の量子ビットの系には$\ket{00}, \ket{01}, \ket{10} \ket{11}$で表される計算基底状態がある。
2個の量子ビットを記述する状態ベクトルは、
\begin{equation}
    \label{eq: QuantumComputationAndQuantumInfomation.1.5}
    \ket{\psi} = \alpha_{00} \ket{00} + \alpha_{01} \ket{01} + \alpha_{10} \ket{10} + \alpha_{11} \ket{11}
\end{equation}
で与えられる。
ここで、$\alpha_{00}, \alpha_{01}, \alpha_{10}, \alpha_{11}$はそれぞれの基底の複素確率振幅である。
単一量子ビットの場合と同様に、測定結果$x(= 00, 01, 10, 11)$は確率$|\alpha_x|^2$で生じ、測定後の量子ビットの状態は$\ket{x}$となる。
確率の合計が1になる条件は正規化状態$\sum_{x \in \{0, 1\}^2} |\alpha_x|^2 = 1$で表される。
ここで、記号「$\{0, 1\}^2$」は「各文字が0または1であり、長さ2の記号列の集合」を意味する。

一般に$n$個の量子ビットを考えると、この系の計算基底は$\ket{x_1 x_2 \cdots x_n}$の形をしており、この系の量子状態は$2^n$個の振幅で規定される。
ここで、$x = x_1 x_2 \cdots x_3$は、$x \in \{ 0, 1 \}^n$であり、$x \in \{ 0, 1 \}^n$は各文字が$0$または$1$であり、長さ$n$の記号列の集合を表す。