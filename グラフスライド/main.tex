% \RequirePackage{plautopatch}
\RequirePackage[l2tabu, orthodox]{nag}

\documentclass[platex,dvipdfmx]{beamer}			% for platex
\usepackage{graphicx}
\usepackage{bxtexlogo}

%デザインの選択 (省略可)
% \usetheme{default}
\usetheme{Frankfurt}

\title{数値計算のグラフまとめ}

\author{9BSP1118 村岡海人}
\date{\today}

% スライドの始まり
\begin{document}

% タイトルページ
\frame{\maketitle}

% スライド Example
\begin{frame}{スライド}
数式を書くことができるよ。
\begin{equation}
    \frac{1}{s^{2}}\frac{\partial^{2} u}{\partial t^{2}} = \frac{\partial^{2} u}{\partial x^{2}} + \frac{\partial^{2} u}{\partial y^{2}} + \frac{\partial^{2} u}{\partial z^{2}}
\end{equation}
\end{frame}

% 目次ページ
\begin{frame}{目次}
    \tableofcontents
\end{frame}

% 目次の具体例
\section{目次の具体例}
\begin{frame}{スライド}
arrayも使えます.
\begin{align}
    x &= a + 3 \\
    y &= b -4
\end{align}
\end{frame}

% 箇条書き
\section{箇条書き}
\begin{frame}{箇条書き}
    \begin{itemize}
        \item item 1
        \begin{enumerate}
            \item item1-1
            \item item1-2
        \end{enumerate}
        \item item 2
        \begin{enumerate}[I]
            \item item 2-1
            \item item 2-2
            \item item 2-3
        \end{enumerate}
    \end{itemize}
\end{frame}

% 表スライド
\section{表}
\begin{frame}{表の追加}
    \begin{table}
        \caption{Caption}
        \label{table:sample}
        \centering
            \begin{tabular}{ccc}
            \hline
            料理    & 値段   &  場所  \\
            \hline \hline
            チキン  & 200円  & 公園 \\
            ピザ    & 300円  & 公園 \\
            ご飯    & 100円  & 室内 \\
            パン    &  70円  &  室内 \\
            \hline
            \end{tabular}
    \end{table}
\end{frame}
\end{document}